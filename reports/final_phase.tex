\documentclass[11pt]{article}
\usepackage{amsmath, amssymb, amsfonts} 
\usepackage{float}
\usepackage{graphics}                 
\usepackage{color}                    
\usepackage{hyperref}
\usepackage{graphicx}
\usepackage{xepersian}                            

\settextfont{XB Niloofar}
%\setlatintextfont{Calibri}
\hypersetup{
	colorlinks=true,
	linkcolor=blue,
	filecolor=magenta,      
	urlcolor=cyan
}
\title{
	بسم الله الرحمن الرحیم
	\\[25pt]
	گزارش پروژه پایانی کارشناسی
}
\author{
یگانه مرشدزاده
	\\[10pt]
	96521488
}
\date{\today} 

\begin{document}
\maketitle
\begin{persian}
	
\newpage
\addtocontents{toc}{\textbf{عنوان}~\hfill\textbf{صفحه}\par}
\tableofcontents
\newpage
\listoffigures

\newpage
\section{
پارامتر E بهتر است فرد باشد یا زوج؟ 
}
    Fast Gradient Sign Method (FGSM)[47]
Projected Gradient Descent (PGD)[48]
Carlini and Wagner (C&W) attack[49]
Adversarial patch attack[50]

\begin{figure}[H]
	\centering{\includegraphics[width=\linewidth, height=\textheight, keepaspectratio]{images/table.png}}
	\caption{جدول به ازای مقادیر مختلف n}
	\label{table}
\end{figure}
به همین علت می‌توان گفت RSA با پیمانه ۱۰۲۴ بیتی، از نظر امنیت معادل با مثلا AES ۸۷ بیتی است.\\


\newpage
\section{منابع}
\begin{flushleft}
	\url{https://en.wikipedia.org/wiki/Adversarial_machine_learning}\\
	\url{https://www.nature.com/articles/d41586-019-03013-5}
\end{flushleft}

\end{persian}
\end{document}          
https://colab.research.google.com/github/google/sentencepiece/blob/master/python/sentencepiece_python_module_example.ipynb#scrollTo=ee9W6wGnVteW