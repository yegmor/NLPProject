\documentclass[12pt, a4paper]{article}
\usepackage{amsmath, amssymb, amsfonts} 
\usepackage{inputenc}
\usepackage{float}
\usepackage{graphics}                 
\usepackage{color}                    
\usepackage{hyperref}
\usepackage{ptext}
\usepackage{graphicx}                         
%\settextfont[Scale=.9]{Calibri}

%\graphicspath{ {images/} }

\hypersetup{
	colorlinks=true,
	linkcolor=blue,
	filecolor=magenta,      
	urlcolor=cyan
}


\title{
	{\Huge \textbf{In the name of God}}
	\\[20pt]
	\includegraphics[width=0.5\linewidth]{../assets/IUST_logo_color_eng.jpg} \\
	{\normalsize Department of Computer Engineering}
}

\author{
	\\[10pt]
	\textbf{{\LARGE Natural Language Processing}}
	\\[10pt]
	\LARGE Final Phase Report
	\thanks{\url{https://github.com/yegmor/NLPProject}}
	
	\\[30pt]
	\textbf{Yeganeh Morshedzadeh}
	\\[5pt]
	Student Number: 96521488
}

\date{Spring 2021} 

\begin{document}
	
\maketitle
%\graphicspath{{../reports/images}}‎
	
\clearpage
%\addtocontents{toc}{\textbf{Content}~\hfill\textbf{Page}\par}
\tableofcontents
\newpage

\listoffigures
\newpage

\listoftables
\newpage

\begin{abstract}
	In this project, we tried to use Natural Language Processing to better understand Depression and Anxiety posts. The dataset is gathered from Reddit communities \href{https://www.reddit.com/r/depression}{r/depression} and \href{https://www.reddit.com/r/Anxiety}{r/Anxiety}.
	\\[10pt]
	
	For this project, at first, we wrote a project proposal (\href{https://docs.google.com/document/d/1tHGEmEgn8-sp8MD72d8NjnZsq-GpVupzsMWgnqaGi-Y/edit?usp=sharing}{Google Docs}), and afterwards, in the first phase (\href{https://docs.google.com/document/d/1Jc2ELhweU01Tbf0WalU7wVQABdAV4w50mhQnmMpU2mM/edit?usp=sharing}{Google Docs}), we gathered data and made some exploratory data analysis. 
	\\[10pt]
	
	In the final phase, we went deeper, and tried various NLP tasks, such as, computing Word2Vec, Tokenization, Parsing, and creating a language model based on our the dataset.
\end{abstract}

\newpage
\part{Word2Vec}
\large{\textbf{Filename:} 3\_word2vec.ipynb}

\section*{Code}
For this part we have three Word2Vec models, named as dep\_w2v\_model, anx\_w2v\_model, and all\_w2v\_model. Moreover, with boolean parameters, load and save, the model will be saved and/or loaded in the my\_word2vec function.

\begin{table}[ht]
	\caption{Word2Vec vocabulary size} 
	\centering 
	\vspace{5mm} 
	\includegraphics[width=0.5\linewidth]{../reports/images/w2v_vocab-size.png}
	\label{table:nonlin} 
\end{table}



\section*{Results and Examples}
To make the visualizations more relevant, we will look at the relationships between a query word (in \textcolor{red}{**red**}), its most similar words in the model (in \textcolor{blue}{**blue**}), and other words from the vocabulary (in \textcolor{green}{**green**})

\newpage
\part{Tokenization}
\large{\textbf{Filename:} 4\_tokenization.ipynb}

\section*{Code}
In this part we have used KFold to split our data into train and test. Afterwards, we train SentencePiece model based on the data. Lastly, we compute <UNK> on our test dataset. 

\section*{Results and Examples}


\newpage
\part{Parsing}
In this part, we used Stanza, which is a a Python NLP Package, and a collection of accurate and efficient tools for the linguistic analysis of many human languages. Starting from raw text to syntactic analysis and entity recognition, Stanza brings state-of-the-art NLP models to languages of your choosing.

More specifically, we used their \href{http://stanza.run/}{Online Demo} to create a manual .CoNLL file based on our dataset. Later, we can use \href{https://universaldependencies.org/conllu_viewer.html}{Universal Dependencies CoNLL viewer} to automatically generate parse tree.

The depen
\newpage
\part{Language Model}
\large{\textbf{Filename:} 5\_language-model.ipynb}

\section*{Code}
In this part we have used KFold to split our data into train and test. Afterwards, we train SentencePiece model based on the data. Lastly, we compute <UNK> on our test dataset. 

\section*{Results and Examples}



\newpage
\part{Fine Tuning}

\section*{Classification}
\large{\textbf{Filename:} 6\_finetune\_classification.ipynb}

\subsection*{Code}
In this part we have used KFold to split our data into train and test. Afterwards, we train SentencePiece model based on the data. Lastly, we compute <UNK> on our test dataset. 

\subsection*{Results and Examples}


\section*{Language Model}
\large{\textbf{Filename:} 7\_finetune\_language-model.ipynb}

\subsection*{Code}
In this part we have used KFold to split our data into train and test. Afterwards, we train SentencePiece model based on the data. Lastly, we compute <UNK> on our test dataset. 

\subsection*{Results and Examples}

%
%\begin{figure}[H]
%	\centering{\includegraphics[width=\linewidth, height=\textheight, keepaspectratio]{images/table.png}}
%	\caption{جدول به ازای مقادیر مختلف n}
%	\label{table}
%\end{figure}


\newpage



\begin{thebibliography}{9}
	\bibitem{bib0}
	\url{https://towardsdatascience.com/goodbye-world-4cc844197d51}
	
	\bibitem{bib1}
	\url{https://colab.research.google.com/github/google/sentencepiece/blob/master/python/sentencepiece_python_module_example.ipynb#scrollTo=ee9W6wGnVteW}
	
	\bibitem{bib2}
	\url{https://gmihaila.github.io/tutorial_notebooks/gpt2_finetune_classification/}
	
	\bibitem{bib3}
	\url{https://colab.research.google.com/github/philschmid/fine-tune-GPT-2/blob/master/Fine_tune_a_non_English_GPT_2_Model_with_Huggingface.ipynb#scrollTo=hKBSyNLgqF9K}
	
	\bibitem{bib4}
	\url{https://github.com/huggingface/notebooks/blob/master/examples/language_modeling.ipynb}
	
	\bibitem{bib5}
	\url{https://www.kaggle.com/pierremegret/gensim-word2vec-tutorial}
	
	\bibitem{bib6}
	\url{https://machinelearningmastery.com/how-to-develop-a-word-level-neural-language-model-in-keras/}
	
	\bibitem{bib7}
	\url{https://huggingface.co/transformers/training.html}
	
	\bibitem{bib8}
	\url{https://www.kaggle.com/achintyatripathi/gensim-word2vec-usage-with-t-sne-plot}
	
	\bibitem{bib9}
	\url{https://colab.research.google.com/github/borisdayma/huggingtweets/blob/master/huggingtweets-demo.ipynb}
	
	\bibitem{bib10}
	\url{https://huggingface.co/transformers/custom_datasets.html}
	
	\bibitem{bib11}
	
\end{thebibliography}

\end{document}          
